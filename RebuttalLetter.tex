% Compile with: pdflatex + bibtex + pdflatex + pdflatex
\documentclass[JiaixnZhong]{CoverLetter}

% use biblatex to manage references
% \usepackage[
%     doi=false,
%     isbn=false,
%     url=false,
%     eprint=false,
%     giveninits=true,
%     date=year,
%     sorting = none,
%     style = numeric-comp
% ]{biblatex}
% \addbibresource{biblatex.bib}

%%%%%%%%%%%%%%%%%%%%%%%%%%%%%%%%%%%%%%%%%%%%%%%%%%%%%%%%%%%%%%%%%%%%%%%%%%%%%%%%
% Set watermark
% \newcommand{\WatermarkImage}{img/penn-state-shield.jpg} % watermark file
% \newcommand{\WatermarkPages}{1-2} % the pages to place watermark
% \usepackage{xwatermark}
% \newwatermark[pages=\WatermarkPages, scale=.5, xpos=0, ypos=0]{
%     \tikz{\node[opacity=0.08]{\includegraphics{\WatermarkImage}}}
% }
%%%%%%%%%%%%%%%%%%%%%%%%%%%%%%%%%%%%%%%%%%%%%%%%%%%%%%%%%%%%%%%%%%%%%%%%%%%%%%%%

\title{
    \vspace{-3em}
    \textbf{\Large Responses to Reviewers' Comments}
    \\[-.5em]
    \textit{\large Re: Manuscript ID 0000-0000-0000}
    \\
    \begin{spacing}{0.8}
        {\large  \textbf{``Your Manuscript Title is Here Your Manuscript Title is Here Your Manuscript Title is Here''} }
    \end{spacing}
    \vspace{-3.5em}
}

\begin{document}

\date{}
\maketitle

% enable Page 1 of xx at the first page
% \thispagestyle{firststyle}
\thispagestyle{firstbottomstyle}
\addcontentsline{toc}{section}{Title}

% \noindent
% \today

% ============================================================
% Comment below to hide the summary of changes table
% ============================================================
\vspace{1em}
We have carefully revised the manuscript and the Supplementary Materials (SM) to address all the reviewers' comments. A summary of the key changes is tabulated below:

\begin{table}[htbp]
  \centering
  \renewcommand{\arraystretch}{1.3}
  \caption{List of major changes in the revised manuscript and Supplementary Materials (SM).
  Rev$i$-C$j$ denotes Comment $j$ from Reviewer $i$.
  All changes are marked in blue in the revised manuscript and SM.
  }
  \vspace{-0.5em}
  \footnotesize
  \begin{tabular}{p{0.09\textwidth} p{0.22\textwidth} p{0.61\textwidth}}
    \toprule
    \textbf{Source} & \textbf{Topic of concern} & \textbf{Revision \& Location} \\
    \midrule
    % Reviewer 1
    \hyperref[sec:R1C1]{Rev1-C1} & Implementation of PBCs/OBCs & \textbf{Added SM Sec.~S2.7 \& Fig.~S11} illustrating the circuit-based boundary switching. Added descriptions in the \textbf{Experimental Section}. \\
    \hline
    \hyperref[sec:R1C2]{Rev1-C2} & Tuning of onsite potentials & \textbf{Added descriptions in SM Sec.~S2.2} detailing the feedback tuning protocol and fitting method. \\
    \hline
    \hyperref[sec:R1C3]{Rev1-C3}, \hyperref[sec:R3C1]{Rev3-C1} & Photo of experimental setup & \textbf{Added Fig.~2 to manuscript} showing the final 56-cavity setup. Updated \textbf{Sec.~2.1} to reference the new figure. \\
    \hline
    \hyperref[sec:R1C4]{Rev1-C4}, \hyperref[sec:R1C5]{Rev1-C5} & Parameter statistics and long-term stability & \textbf{Performed long-term stability experiments (6-hour continuous test).} \textbf{Added SM Sec.~S2.5 \& Figs.~S5--S10} presenting stability data and distribution statistics. Added summary in \textbf{Sec.~2.1}. \\
    \hline
    \hyperref[sec:R1C6]{Rev1-C6} & Chern number calculation & \textbf{Expanded SM Sec.~S1} to include the discrete summation formula and convergence details. \\
    \hline
    \hyperref[sec:R1C7]{Rev1-C7} & Experimental observation of EPs & \textbf{Conducted new experiments on a pristine $4\times4$ lattice.} Measured spectral coalescence, phase rigidity dip, and line-gap closing. \textbf{Added Fig.~3} and \textbf{Sec.~2.4} to the manuscript.
    Added \textbf{SM Sec.~S4}, \textbf{SM Sec.~S5}, and Figs.~S14--S19 for detailed analysis.
    Revised \textbf{Abstract}.
     \\
    \hline
    \hyperref[sec:R1C8]{Rev1-C8} & Device applications & \textbf{Revised Conclusion} to outline principles of NHDS-based concentrators and D-NHSE-based sensors. \\
    \hline
    % Reviewer 2
    \hyperref[sec:R2C1]{Rev2-C1} & Novelty and comparison with Wu \textit{et al.} \cite{Wu2025ObservationDislocationNonHermitian} & \textbf{Added Note Added} at the end of the manuscript and clarification in \textbf{Introduction} distinguishing line-gap vs. point-gap physics. \\
    \hline
    \hyperref[sec:R2C2]{Rev2-C2} & Advantage of active coupling vs. circuits & \textbf{Revised Sec.~2.1} to justify the use of active meta-atoms for realizing time-reversal breaking and wave-based Hamiltonians. \\
    \hline
    \hyperref[sec:R2C3]{Rev2-C3} & General applicability of Green's function method & \textbf{Performed validation experiments on a passive-tube system.} \textbf{Added SM Sec.~S2.8 \& Fig.~S12} demonstrating the successful extraction of complex spectra. \\ 
    \hline
    % Reviewer 3
    \hyperref[sec:R3C2]{Rev3-C2} & Amplifier details \& self-oscillation & \textbf{Added SM Sec.~S2.6} identifying audio amplifier usage. Added text to \textbf{Experimental Section} on stability verification via eigenenergy imaginary parts. \\
    \hline
    \hyperref[sec:R3C4]{Rev3-C4} & Spectral winding vs. D-NHSE & \textbf{Revised Introduction} (2nd para.) to clarify the distinction between point-gap and line-gap D-NHSE. \\
    \hline
    \hyperref[sec:R3C5]{Rev3-C5} & Raw spectral responses & \textbf{Added SM Sec.~S3 \& Fig.~S13} showing raw point-to-point Green's functions. Added descriptions in \textbf{Sec.~2.2}. \\
    \bottomrule
  \end{tabular}
  \label{tab:revision_summary}
\end{table}
% ============================================================


% ============================================================ 
% Comment from Editor
% ============================================================ 
\section{Comments from Editor}
\begin{commentbox}
    \lipsum[1]
\end{commentbox}

\subsubsection*{Responses}
\lipsum[2-3]

\subsubsection*{Changes}
\lipsum[3]


\newpage
\section{Comments from Reviewer 1}

\subsection{Comment 1\label{sec:R1C1}}
\begin{commentbox}
    Here are the general comments from Reviewer 1.
\end{commentbox}


\subsubsection*{Responses}
Thank you for your comments and recommendation.
Reference~\cite{Zhong2022QuietZoneGeneration, Zhong2020InsertionLossThin,Zhong2020SphericalExpansionAudio}.
Figure~\ref{fig:ex} is an example image \cite{Zhong2020SphericalExpansionAudio}.

\begin{figure}[!htb]
    \centering
    \includegraphics[width = 0.4\textwidth]{example-image}
    \caption{Example image.}
    \label{fig:ex}
\end{figure}

\subsection{Comment 2\label{sec:R1C2}}
\begin{commentbox}
    \lipsum[3]
\end{commentbox}

\subsubsection*{Responses}
We added Sec.~IV.C entitled ``Computational efficiency'' in the revised manuscript to compare the calculation time of convolution models and the exact solution.
Equation~(\ref{eq:ex}) is the expression.

\begin{equation}
    f(x) = x^2 + 1.
    \label{eq:ex}
\end{equation}

\subsubsection*{Changes}
\begin{itemize}
    \item At line 405,
          Sec.~IV.C entitled ``Computational efficiency'' is added.
    \item Table~\ref{tab:2}.
\end{itemize}


\begin{table*}
    \caption{The position and weight coefficients for the uniform and optimal array configurations.}
    \label{tab:2}
    \centering
    \begin{tabular}{ccccc}
        \toprule
        \multirow{2}{4em}{Element index, $n$ }
          & \multicolumn{2}{c}{Position, $x_n$ (mm)}
          & \multicolumn{2}{c}{Weight coefficients, $w_n$}
        \\
          & Uniform array                                  & Optimal array
          & Uniform array                                  & Optimal array            \\
        \midrule
        1 & $-60$                                          & $-60$         & 1 & 1.99 \\
        2 & $-43$                                          & $-49$         & 1 & 1.17 \\
        8 & 60                                             & 60            & 1 & 1.51 \\
        \bottomrule
    \end{tabular}
\end{table*}



\clearpage
\section{Comments from Reviewer 2}

\subsection{General Comments}
\begin{commentbox}
    \lipsum[6]

    \lipsum[7]
\end{commentbox}

\subsubsection*{Responses}
Thank you for the suggestion.

\subsubsection*{Changes}
\begin{itemize}
    \item Line 123, change 1.
    \item Line 234, change 2.
\end{itemize}


\subsection{Comment 1\label{sec:R2C1}}
\begin{commentbox}
    \lipsum[8]
\end{commentbox}
\subsubsection*{Responses}
We appreciate the comment.

\subsection{Comment 2\label{sec:R2C2}}
\begin{commentbox}
    \lipsum[8]
\end{commentbox}
\subsubsection*{Responses}
We appreciate the comment.

\clearpage

% for bibtex
\bibliographystyle{unsrt}
% bibtex.bib is the main reference file. Please add your references there.
% bibtex_Jiaxin.bib is an additional reference file for Jiaxin Zhong's personal references.
\bibliography{bibtex.bib, bibtex_Jiaxin.bib}

% for biblatex
% \printbibliography 

\addcontentsline{toc}{section}{References}
\end{document}
